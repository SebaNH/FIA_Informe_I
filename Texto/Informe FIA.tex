\documentclass[10pt]{article}
\usepackage{geometry}                % See geometry.pdf to learn the layout options. There are lots.
\geometry{letterpaper}                   % ... or a4paper or a5paper or ... 
\usepackage{graphicx}			% insertar gráficos
\usepackage{amssymb}			% Símbolos
\usepackage{hyperref}			% para incluir hiper-referencias
\usepackage{titlesec}			% Cambiar formato de títulos
\usepackage[plain, noabstract, nocomment]{flexbib} % Para citar con paréntesis y cosas así 
% https://www.nacho-alvarez.es/index.php/blog/2007/04/15/estilo-de-bibliografia-para-bibtex/
% http://www.latex.um.es/retazos/leccion_15/flexbib_manual.pdf
% Editato el spanishbst.tex y flexbib.sty
\usepackage{appendix} % Anexos
\usepackage{subfigure} % Para incluir varias figuras
\usepackage{multirow}
\usepackage{fontspec,xltxtra,xunicode}		% Por defecto
\defaultfontfeatures{Mapping=tex-text}		% Para usar Century Gothic
\setmainfont[		% Para usar Century Gothic
 BoldFont={Century Gothic Bold}, 
 ItalicFont={Century Gothic Italic},
 BoldItalicFont={Century Gothic Bold Italic}
 ]{Century Gothic}

\bibpunct{(}{)}{;}{a}{,}{,} % Modificar como se cita https://en.wikibooks.org/wiki/LaTeX/Bibliography_Management

\renewcommand{\bibsection}{\section{Bibliografía\\}} % Cambiar nombre de la bibliografía insertada al final del documento
\renewcommand{\figurename}{Figura} % Cambiar el nombre de las figuras
\renewcommand{\tablename}{Cuadro} % Cambiar el nombre de las tablas
\renewcommand{\contentsname}{Tabla de contenidos\\} % Cambiar el nombre de la tabla de contenidos
\renewcommand{\listfigurename}{Índice de figuras\\} % lista de figuras
\renewcommand{\listtablename}{Índice de cuadros\\} % lista de cuadros
%http://www.elmundoenbits.com/2012/03/latex-problemas-con-el-anexos.html#.WOrWdVKZNE4
\renewcommand{\appendixname}{Anexos}
\renewcommand{\appendixtocname}{Anexos}
\renewcommand{\appendixpagename}{Anexos}


\titleformat*{\section}{\normalsize\bfseries}
\titleformat*{\subsection}{\normalsize\bfseries}
\titleformat*{\subsubsection}{\normalsize\bfseries}

\title{Cuenca del Río Elqui}
\author{Erick Millón}
%\date{}                                           % Activate to display a given date or no date

\begin{document}
\maketitle

\section{Descripción General.}
La cuenca del Río Elqui está ubicada en la provincia de Elqui, Región de Coquimbo, entre los paralelos 29°34’ – 30°27’ Latitud Sur y meridianos 71°22’ – 69°52’ longitud Oeste. Al norte limita con las cuencas del Río Huasco, quebrada de Los Choros, Honda y Chacay, al Sur con la cuenca del Río Limarí y las cuencas  costeras de las quebradas El Culebrón y Lagunillas, al Este con la Republica Argentina y al Oeste con el Océano Pacifico (Dattwyler, 2008).
Esta cuenca presenta un régimen nivo-pluvial, vale decir que el caudal se origina por el derretimiento de la nieve acumulada en la alta cordillera, así como del escurrimiento de las precipitaciones en forma de lluvia de las subcuencas de menor cota. Además existen afloramientos y vertientes que se producen a lo largo de su lecho, así como derrames y sobrantes devueltos al río por los canales de riego. En invierno el escurrimiento es principalmente de origen pluvial  mientras que en primavera y verano es de origen nival (Galleguillos, 2004).
El principal cauce de esta cuenca y que le da el nombre a la misma corresponde al Río Elqui, el cual se forma por la confluencia de los ríos Turbio y Claro, en las cercanías de la localidad de Rivadavia, aproximadamente a 75 km de La Serena. Las principales quebradas aportantes a este cauce natural ubicadas en la ribera norte corresponden a Marquesa y Santa Gracia. En la ribera sur destacan las quebradas de San Carlos, “Arrayan” y Talca (DGA, 2004).

\subsection{Hidrografía.}
Los principales tributarios del río Elqui son los ríos Turbio y Claro, aguas debajo de la unión de dichos cauces se habla de río Elqui, este punto se ubica a 815 m.s.n.m en la cercanías de la localidad de Rivadavia (figura xxx, mapa con ríos de Elqui).

\subsubsection{Río Turbio.}
La cuenca del río turbio posee una superficie total de 4.196 $km^2$. Este cauce se forma 43 km aguas arriba de la localidad de Rivadavia a 1.370 m.s.n.m, a partir de la unión de los ríos Toro y La Laguna. Ambos cauces tienen su origen en el área norte de la cordillera en los límites con la República Argentina. El río Toro drena la zona nororiente y sus principales tributarios corresponden al estero Tambo el que cambia de denominación a río Vacas Heladas y los ríos malo y Toro Muerto. El Río La Laguna se ubica al sur de la cuenca del Río Toro y en su cabecera se ubica el único glaciar que existe en la cuenca, el glaciar El Tapado. Los principales cauces tributarios al rio la laguna corresponden a los ríos Colorado y La Gloria. (figura xxx, mapa con ríos de turbio) El principal tributario al río Turbio corresponde al río Incahuaz, lo que ocurre en el sector de las terneras. Este rio al igual que los ríos Toro, La laguna y Turbio tiene un régimen marcadamente nival, lo que hace que el régimen de escurrimiento sea permanente. Otro cauce de importancia afluente al río Turbio corresponde a la Quebrada del Calvario, cuya cuenca se ubica al norte del río Turbio y con respaldo principalmente pluvial. A partir de dicho punto a la altura de la localidad de Huanta, el río Turbio cambia de rumbo a uno final N-S que es la prolongación del rumbo que trae la quebrada tributaria del Calvario (Zavala et al. 2008)

\subsubsection{Río Claro.}
El río Claro se forma de la unión del río Cochiguaz y el Estero Derecho, en la localidad de Montegrande a 1.223 m.s.n.m. La subcuenca del río Cochiguaz, colinda con la subcuenca del río La Laguna y su nacimiento es en la alta cordillera en zonas vecinas a la República Argentina y su único afluente es el rio Cochiguaz. El estero Derecho limita por el sur con la cuenca del río Hurtado. Aguas debajo de la localidad de Montegrande el río claro recibe como principal aporte a la Quebrada de Paihuano, a la altura de la localidad del mismo nombre. (figura xxx, mapa con ríos de claro)
El área total de los ríos Turbio y Claro, esto es el área tributaria  al río Elqui desde donde comienza dicha denominación propiamente tal es de 5.719 $km^2$, lo que corresponde al 58,4\% del  área total de la cuenca. Esto significa que desde este punto hacia aguas abajo se ubica el 41,6\% de la cuenca, la que de acuerdo a las altitudes que la conforman hacen que sea formado por una serie de quebradas de régimen pluvial. La dirección de este curso de agua es claramente Este-oeste, con una distancia total hasta su desembocadura de aproximadamente 65 km. Por la ribera norte las quebradas más importantes son (desde aguas arriba hacia aguas abajo): Marquesa, Los Perales y Santa Gracia, que de acuerdo a Morales (2001) poseen superficies de 939,8 $km^2$, 39,8 $km^2$ y 1.066,8 $km^2$ respectivamente. Por la ribera sur las principales quebradas tributarias son (desde aguas arriba hacia aguas abajo): San Carlos, El Arrayán, Talca y Las Animas, con superficies de 252 $km^2$, 571,8 $km^2$, 146 $km^2$ y 45,5 $km^2$ respectivamente. La totalidad de estas subcuencas conforman un área de 3.061,8 $km^2$, lo que corresponde al 32,5\% del área total de la cuenca del Río Elqui y al 83,2\% dela rea de la cuenca aguas debajo de la junta de los ríos Turbio y Claro (Zavala et al. 2008). (figura xx, mapa con quebradas de Elqui)

\subsection{Clima.}
En el caso de la Región de Coquimbo, un factor dominante del clima es el Anticiclón del Pacifico que provoca el permanente bloqueo de los sistemas frontales provenientes del S-W, causante de lluvias. La presencia del anticiclón da forma al carácter semiárido de la zona. Sin embargo, durante los años de ocurrencia del Fenómeno del Niño (Ciclo ENSO), el anticiclón se desplaza hacia el norte, y la región puede recibir precipitaciones 2 o 3 veces superiores a lo normal. Por otra parte, pese a que las precipitaciones en la desembocadura de la Cuenca son del orden de solo 90 mm en años normales, ellas llegan a duplicarse en la cabecera de la cuenca, lo que determina efectivamente el caudal de los ríos (Galleguillos, 2004).
Por otra parte, el relieve propio de la cuenca posee gran importancia en las características de su clima, ya que su forma controla el ingreso hacia tierras interiores de las masas de aire húmedo y de los escasos sistemas frontales que alcanzan estas latitudes. 
La Corriente de Humboldt tiene un efecto moderador del régimen térmico, estabilizador del aire y sobre la tasa de evaporación del agua, limitando la formación de nubes que generan precipitaciones. Finalmente la fisiografía de la región, controla la intrusión de masas de aire marino que transportan la neblina costera hacia los valles (Galleguillos, 2004).
La cuenca del Río Elqui, presenta tres tipos climáticos, el Estepárico costero o nuboso, Estepa Cálido y Templado Frio de Altura (DGA, 2004).

\subsubsection{Clima Estepárico costero o Nuboso.} 
Se presenta a lo largo de toda la costa. Su influencia llega hasta 40 km al interior, por medio de los valles transversales y quebradas. Su mayor característica es la abundante nubosidad; humedad, temperaturas moderadas, con un promedio de precipitaciones de 130 mm anuales con un periodo seco de 8 a 9 meses.

\subsubsection{Clima Estepa Cálido.}
Este clima se sitúa en el valle del río Elqui, por sobre los 800 m.s.n.m. hasta los 3.000 m.s.n.m. y se caracteriza por la ausencia de nubosidad y sequedad del aire. Sus temperaturas son mayores que en la costa. A medida que nos adentramos en el Valle del Elqui, las temperaturas aumentan alcanzando las máximas a los 1.200 m.s.n.m., luego comienzan a disminuir nuevamente. En invierno se presentan intensan heladas. Las precipitaciones no son tan abundantes y los periodos de sequía son característicos. Las precipitaciones promedio anuales son de 90 mm, solo aumentadas en los años con presencia del fenómeno de El Niño, oportunidad en que esta cifra puede duplicarse o triplicarse.

\subsubsection{Clima Templado Frío de Altura.}
este clima se localiza en la Cordillera de Los Andes sobre los 3.000 metros de altitud con características de altas precipitaciones, temperaturas bajas y nieves permanentes que constituyen un aporte significativo de agua en el periodo estival. El hecho de que la alta cordillera registre una pluviosidad mayor, favorece la acumulación de nieve en invierno y las reservas hídricas naturales para la época estival. En algunas temporadas las áreas cordilleranas presentan precipitaciones en la temporada estival, producto de tormentas tropicales que se desplazan al sur desde el altiplano boliviano
Con relación a las precipitaciones registradas en toda la cuenca, los registros de precipitación media anual corresponden a 73,9 mm en el sector de Huanta, 92,4 mm en Paihuano y 137,5 mm en la localidad de Vicuña. El total de agua caída por año alcanza a 125,7 mm. 
En el caso de las temperaturas registradas en la cuenca, se considera la caracterización descrita  en el estudio “Zonificación Agroclimática de la IV región” (DMC, 2016) donde a partir de estaciones agroclimáticas emplazadas en la región se logra describir las variaciones climáticas que en esta se registran. En la Cuenca del río Elqui, este estudio considero 4 estaciones: La Serena, Pan de Azúcar, Vicuña y Pisco Elqui, las cuales logran describir la amplitud de regímenes térmicos registrados en la zona de estudio. 

\begin{itemize}
\item En el caso de la estación ubicada en La Serena, se caracteriza por presentar una temperatura media anual de 14,3°C, con una máxima media del mes más cálido (enero) de 21,4°C, valor que está influenciado por la cercanía al mar, cumpliendo un rol moderador del régimen térmico. 
\item La estación ubicada en el sector de Pan de Azúcar, presenta una temperatura media anual de 14,1°C, con una máxima media en enero de 24,4°C y una temperatura mínima promedio en julio de 5,4°C. Presenta un periodo libre de heladas de 8 meses entre agosto a marzo. 
\item La estación ubicada en la ciudad de Vicuña registra un régimen térmico donde la temperatura media anual de 15,2°C, con una máxima media del mes más cálido (enero) de 29°C y una temperatura mínima promedio del mes más frío (julio) de 5,2°C. el periodo libre de heladas se extiende entre los meses de septiembre a marzo.
\item La estación ubicada en la localidad de Pisco Elqui, presenta un régimen térmico donde la temperatura media anual alcanza los 16,1°C, estableciéndose una temperatura máxima promedio del mes más cálido (enero) de 32,7°C y una mínima media del mes más frío (julio) de 6,5°C.
\item En la parte alta de la Cordillera de Los Andes son frecuentes las temperaturas inferiores a 0°C, lo que hace que las precipitaciones invernales sean preferentemente sólidas.
\end{itemize}

Finalmente toda la caracterización climática previa permiten definir para la Cuenca del río Elqui un  periodo seco que se manifiesta entre los meses de Septiembre a Abril con precipitaciones medias mensuales que varían entre 0 y 3,1 mm y temperaturas comprendidas entre 13,1 y 20,1 °C (enero). El periodo húmedo, se presenta desde Mayo a Agosto registrando precipitaciones mensuales entre 18,8 y 27,6 mm y temperaturas entre 11,5 y 13,6 °C (DGA, 2004).

\section{Organizaciones involucradas en la administración del recurso hídrico en la cuenca del Río Elqui.}
En la cuenca del río Elqui existen dos organizaciones que administran en forma directa los recursos hídricos; la principal organización es la Junta de Vigilancia del Río Elqui y sus afluentes (JVRE) y la Junta de Vigilancia del Estero Derecho (JVED)
El primer Rol de regantes existente en la cuenca fue la Asociación de Canalistas del Río Coquimbo y sus Afluentes, conformada en junio de 1943 y contemplaba 191 bocatomas o encauzamientos.  El Estero Derecho, como tributario solo ocasional del sistema, formó una Junta de Vigilancia independiente en 1968. El 11 de Junio de 1993 se constituye la Junta de Vigilancia del Río Elqui y Sus Afluentes remplazando a la anterior Asociación de Canalistas del Río Coquimbo. A la fecha la Junta del Río Elqui posee bajo su jurisdicción 117 canales más 6 captaciones a través de elevación mecánica (Morales, 2005).

\subsection{Junta de Vigilancia Estero Derecho.}
Por Decreto Supremo N°26, del 10 de enero de 1977, del Ministerio de Obras Públicas, se aprobó la constitución y estatutos de la Junta de Vigilancia del Estero Derecho, y posteriormente mediante Resolución Exenta N°2.568, de 20 de Septiembre de 2001, de la Dirección General de Aguas, se dispone su anotación en el Registro de Juntas de Vigilancia y se la declara organizada.
La Junta de Vigilancia ejerce jurisdicción en el estero Derecho que es afluente del río Claro, tributario por la izquierda del río Elqui y la Quebrada de Paihuano. 
Los canales que forman parte de esta Junta de Vigilancia, que alcanza a un total de 20 canales, se encuentran legalmente organizados, como Comunidades de Aguas e inscritas todos en el Conservador de Bienes Raíces (CBR) de Vicuña (CNR, 2013).

\subsection{Junta de Vigilancia del Río Elqui y sus afluentes.}
Por Decreto Supremo N° 173, de 11 de junio de 1993, del Ministerio de Obras Públicas, se aprobó la constitución y estatutos de la Junta de Vigilancia del Elqui y sus afluentes, y posteriormente mediante Resolución Exenta N°1.606, de 25 de Junio de 1996, de la Dirección General de Aguas, se dispone su anotación en el Registro de Juntas de Vigilancias y se la declara Organizada. La Junta de Vigilancia ejerce jurisdicción en el río Elqui y sus afluentes y subafluentes los ríos Claro, Turbio y Cochiguaz.
La mayor parte de los canales que forman parte de esta Junta de Vigilancia se encuentran organizados, casi todos ellos como Comunidades de Aguas y otras pocas como Asociaciones de Canalistas; otros tantos canales funcionan como organizaciones de hecho. Finalmente existen aquéllos de uso exclusivo, esto es, que pertenecen a un solo usuario (CNR, 2013).
El sistema de riego del Río Elqui y sus afluentes está compuesto por dos embalses conectados entre sí (La Laguna y Puclaro) y de una extensa red de 126 canales de riego que captan sus aguas por medio de 117 bocatomas y captaciones de elevación mecánica. Las aguas superficiales de donde extraen gravitacionalmente los canales corresponde a los ríos Turbio, Cochiguaz, Claro y Elqui específicamente (Zavala, 2008).

\section{Estructura administrativa de la Junta de Vigilancia del Río Elqui.}
La Junta está encabezada por la Asamblea General de accionistas, el cual elige un Directorio conformado por 7 Directores que según los estatutos de la organización permanecen durante 3 años en su cargo con la posibilidad de ser reelegidos indefinidamente. El Directorio es asesorado técnicamente por el Ingeniero Repartidor de Aguas y el Subdelegado del Río y apoyado administrativamente por un encargado de finanzas y una secretaria (figura xx). A su vez se cuenta con asesores externos en temáticas jurídicas, contables e informáticas (CNR, 2007).

\section{Administración del recurso hídrico JVRE.}%ingresar figura%
La Junta de Vigilancia del Río Elqui, reconoce una división interna del área jurisdiccional del río en tres secciones (figura xx), las cuales se refieren a sectores de distribución de agua en forma homogénea.

\subsection{Primera sección.}
Agrupa a todos los canales cuyas bocatomas se encuentran adyacentes a los ríos Cochiguaz y Claro. Esta sección presenta un total de 3.855,28 acciones brutas y 469 regantes, los cuales se abastecen del recurso hídrico a través de 39 puntos de entrega, divididos en 3 captaciones y 36 canales.

\subsection{Segunda sección.}
Agrupa a todos los canales cuyas bocatomas se encuentran adyacentes a los ríos Turbio y Elqui, este último sobre embalse Puclaro. Esta sección presenta un total 8.238,87 acciones brutas y 1.696 regantes, los cuales se abastecen del recurso hídrico a través de 53 puntos de entrega, divididos en 12 captaciones y 41 canales.

\subsection{Tercera sección.}
Agrupa a todos aquellos canales cuyas bocatomas se encuentran adyacentes al río Elqui bajo  embalse Puclaro. Esta sección presenta un total de 13.234,16 acciones brutas y 1.342 regantes los cuales se abastecen del recurso hídrico a través de 17 puntos de entrega, todos ellos corresponden a canales.

\section{Infraestructura para la distribución de las Aguas (JVRE).}
La distribución de las aguas en la cuenca del Elqui se lleva a cabo mediante dos embalses y una red de canales de 600,11 km de longitud.

\subsection{Embalse La Laguna.} %insertar figura%
El embalse La Laguna se encuentra ubicado en la zona cordillerana a 3.275 m.s.n.m distante a 100 kilómetros al oriente de la ciudad de Vicuña (figura xx). Este embalse presenta una cortina de concreto armado en la parte inferior y recubrimiento de enrocado. Tiene capacidad para almacenar 40 millones $m^3$. Las aguas del embalse son entregadas al río La Laguna para ser conducidas por el río Turbio hasta el río Elqui, donde son captadas por los diversos canales existentes. Su construcción data de los años 30 y su entrada en operación desde el año 1941  (CNR, 2007).

\subsection{Embalse Puclaro.}%insertar figura%
El embalse Puclaro se encuentra ubicado en el Valle de Elqui, a 46 km al oriente de la ciudad de La Serena a 432 m.s.n.m. (figura xx).
En marzo de 1996 se aprueba su construcción y en agosto del mismo año se inician las obras que permiten que el día 15 de octubre de 1999 puedan iniciar el llenado del embalse. Los estudios determinaron que la capacidad de almacenamiento de 207.000.000 $m^3$.
El embalse de generación multianual regula el Río Elqui, permitiendo una adecuada seguridad de riego a 20.700 ha aproximadamente, lo que significa más que duplicar las áreas regadas antes de la construcción (Dattwyler, 2008).

\subsection{Red de Canales.}
La red de canales dependientes de la JVRE, está compuesta por 107 bocatomas, que a su vez presentan igual número de canales de aducción y que abastecen a 109 puntos de entregas. Existen 10 canales de aducción unificados, vale decir que abastecen a más de un punto de entrega, por esta razón existe una diferencia entre el número de bocatomas y puntos de entrega. De esta manera el número definitivo de entregas administrados por la JVRE corresponde a 109, de los cuales 15 corresponden a captaciones y 94 a canales (PROMMRA, 2016).
La longitud total de los canales actualmente administrados por la JVRE corresponde a 600,11 km (cuadro xx), donde el 57,8\% de la longitud de canales se encuentra en la sección río Elqui bajo Puclaro. Sin embargo, la longitud en uso de dichos canales corresponde solo a  519,43 km, dicha extensión se encuentra en un 78,4\% sin revestimiento (cuadro xxx).\\

\begin{table}[]
\centering
\caption{Longitud de canales por sección en la cuenca del río Elqui, bajo administración de la JVRE.}
\label{my-label}
\begin{tabular}{|c|c|c|}
\hline
\textbf{Seccíon cuenca SIMCA}   & \textbf{Seccíon cuenca JVRE} & \textbf{Longitud (km)} \\ \hline
Río Turbio                      & 2a                           & 30,72                  \\ \hline
Río Cochiguaz                   & 1a                           & 28,82                  \\ \hline
Río Claro                       & 1a                           & 27,55                  \\ \hline
Río Elqui Sobre Embalse Puclaro & 2a                           & 166,01                 \\ \hline
Río Elqui Bajo Embalse Puclaro  & 3a                           & 347,00                 \\ \hline
\multicolumn{2}{|c|}{\textbf{Total}}                           & \textbf{600,11}        \\ \hline
\end{tabular}
\end{table}

\begin{table}[]
\centering
\caption{Longitud de revestimiento según sección de los canales administrados por la JVRE.}
\label{my-label}
\begin{tabular}{|c|c|c|c|c|c|c|}
\hline
\multirow{2}{*}{\textbf{\begin{tabular}[c]{@{}c@{}}Seccíon \\ SIMCA\end{tabular}}} & \multirow{2}{*}{\textbf{\begin{tabular}[c]{@{}c@{}}Sección \\ JVRE\end{tabular}}} & \multirow{2}{*}{\textbf{\begin{tabular}[c]{@{}c@{}}Sin Revestimiento \\ (km)\end{tabular}}} & \multicolumn{3}{c|}{\textbf{Revestimiento (km)}}                                                               & \multirow{2}{*}{\textbf{\begin{tabular}[c]{@{}c@{}}Total\\ (km)\end{tabular}}} \\ \cline{4-6}
                                                                                   &                                                                                   &                                                                                             & \textbf{1/3}                       & \textbf{2/3}                       & \textbf{100\%}                       &                                                                                \\ \hline
Río Turbio                                                                         & 2a                                                                                & 27,03                                                                                       & 0,37                               & 0,16                               & 4,86                                 & \textbf{32,41}                                                                 \\ \hline
Río Cochiguaz                                                                      & 1a                                                                                & 15,54                                                                                       & -                                  & -                                  & 9,54                                 & \textbf{25,08}                                                                 \\ \hline
Río Claro                                                                          & 1a                                                                                & 20,61                                                                                       & 0,06                               & 0,36                               & 8,44                                 & \textbf{29,47}                                                                 \\ \hline
Río Elqui sobre Puclaro                                                            & 2a                                                                                & 101,97                                                                                      & 0,04                               & 0,01                               & 43,11                                & \textbf{145,14}                                                                \\ \hline
Río Elqui bajo Puclaro                                                             & 3a                                                                                & 241,88                                                                                      & 0,22                               & 0,14                               & 45,07                                & \textbf{287,32}                                                                \\ \hline
\multicolumn{2}{|c|}{\textbf{Total}}                                                                                                                                   & \textbf{407,05}                                                                             & \multicolumn{1}{l|}{\textbf{0,69}} & \multicolumn{1}{l|}{\textbf{0,67}} & \multicolumn{1}{l|}{\textbf{111,02}} & \multicolumn{1}{l|}{\textbf{519,43}}                                           \\ \hline
\end{tabular}
\end{table}

\begin{thebibliography}{X}
	
	
	\bibitem{COMISIÓN NACIONAL DE RIEGO (CNR). 2006. 
	\textit{Programa de “Transferencia Tecnologías de Riego/Validación  Sistemas Productivos, Puclaro-Elqui, IV Región, II Etapa”.}  	    \textsc{[En línea] Disponible en: < http://bibliotecadigital.ciren.cl/bitstream/handle/123456789/9486/CNR-0091_1.pdf?sequence=1&isAllowed=y >.} {consulta: 06 marzo 2018}.
	
    %\bibitem{Alvarez} \textsc{Alvarez, P.}, \textsc{Kretshmer, N.} y \textsc{Oyarzún, R.}. \textsc 2006.\textit {Management for Irrigation in Chile: Causes and Consequences. Technology, resource Management and Development, Wasser Berlin}.\textsc {Disponible en: http://www.tt.fhkoeln.de/publication/ittpub\%20303101\_11.pdf.}
    	
	\bibitem{Morales} \textsc{Morales, C.},\textsc {Rojas, R.}2010.
	\textit{Análisis del manejo operacional para escenarios críticos del embalse La Paloma}.Seminario de Titulo, La Serena, Chile. Facultad de Ingenieria. Universidad de La Serena.198p.
		
	\bibitem{Vivanco} \textsc{Vivanco, C.2014}.\textit {Operatividad del Sistema Paloma en base a criterios de asignación hídrica, aplicado al modelo WEAP.} \textsc Seminario de Titulo, Ovalle, Chile. Facultad de Ciencias. Universidad de La Serena. 200p.
	
\end{thebibliography}

\end{document}  